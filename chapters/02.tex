\chapter{Historický přehled generací}

V průběhu 20. století a přelomu tisíciletí se vlivem světových událostí, technologickým rozkvětem a společenskými změnami utvářeli generace s charakteristickými rysy a rozdílným hodnotovým žebříčkem. Generací chápeme \textbf{skupiny lidí se společnými znaky narozenou či vyrůstající v určitém časovém rozmezí}. Sdílené prožitky tvarují a charakterizují osobnosti jedinců narozených v téže generaci.
Je důležité, aby si organizace byly vědomy těchto odlišností a rozdílných hodnotových žebříčků a jednali tak v jejich souladu a tvarovali se dynamicky s přicházejícími generacemi, které nahrazují ty předešlé. S novou pracovní silou přicházejí nové požadavky a potřeby.

V této kapitole budou uvedeny a stručně charakterizovány generace 20. století až po současnost. V další kapitole bude poté pozornost věnována generacím X a Y, které jsou předmětem této práce. Tyto generace momentálně zastávají největší část pracovního trhu\footnote{Dle časopisu Forbes je pouze Generace X v USA zodpovědná za více než třetinu státního příjmu.\cite{forbes2019generationX}}.

\section{Generace G.I.}
Tuto poválečnou generaci, nazývanou také jako \uv{The Greatest Generation}, jak ji nazval Tom Brokaw ve své stejnojmenné knize z roku 1998\cite{cnn2013americangenerationfacts}, tvoří skupina jedinců, kteří se narodili mezi lety \textbf{1900 až 1924}. Tito lidé tedy zažili \textbf{velkou depresi}\footnote{Z anglického \textit{\uv{The Great Depression}}, byla největší ekonomickou recesí moderní historie a nejkatastrofičtější událostí 20. století\cite{investopedia2019thegreatdepression}} i druhou světovou válku.\\
V americkém pojetí je tato generace tzv. \uv{generací \textit{federace}}. Její členové se zasloužili o vybudování Spojených Států, jak je známe dnes. Mezi nejznámnější patří tehdejší americký prezident J. F. Kennedy a další prezidenti: Richard M. Nixon, Gerald R. Ford, Ronald Reagan, Jimmy Carter, and George H.W. Bush. Děti této generace jsou známé jako generace \uv{baby boomers} (viz sekce \ref{sec:baby-boomers}).

\section{Tichá generace}
Označení \uv{tichá} odkazuje na \textbf{konformistické tendence}\cite{bergh2012coolznacky}, tj. přizpůsobování se převažujícím názorům a podřízení se společnosti a režimu nebo částečné potlačení vlastní identity. Vzhledem k časovému úseku, pro které je tato generace definována --- \textbf{1928 až 1945} --- tato konformita není příliš překvapivá. Tichá generace bývá označována za protiklad ke generaci baby boomu.


\section{Baby boomers}\label{sec:baby-boomers}
Označení generace baby boomu vychází z \textbf{nárůstu porodnosti} po skončení druhé světové války, který začal po jejím skončení a trval \textbf{až do roku 1964} (komerční uvedení antikoncepce).\cite{bergh2012coolznacky} Tato generace zažívá technologický pokrok --- --- a je pro ni tedy charakteristická velká flexibilita a přizpůsobivost.
V současnosti tato generace zastává asi 20\rm \% celkové americké populace a zůstává tak zatím \textbf{největší generací} americké historie\cite{investopedia2019babyboomers} (toto se má změnit s příchodem mileniálů).


\section{Generace X}\label{sec:generace-x}
Podíváme-li se okolo sebe, pravděpodobně je okolo nás nějaký \textit{Gen X'er}, narozen v letech \textbf{1965 až 1979 (1981)}. Tato generace je často označována za generaci lenochů (slackers) a vyznačuje se lhostejností. Označení \textit{X} popularizovala kniha Douglase Couplanda \textit{Generace X}, ve které se generace nesnášející nálepky uvádí \uv{říkejte nám prostě X}.\cite{bergh2012coolznacky}
Členové této generace začínali kariéru v 90. letech, které se vyznačovaly ekonomickou recesí a snižováním stavů (a s výjimkou několika vzestupů tento stav alternoval prakticky až do roku 2010\cite{kotler2009chaotika}). Generace X ale převzala pracovní etiku a nasazení generace baby boomu a postupně začala utvářet ekonomiku (viz samostatná kapitola). Podle časopisu Forbes patří generace X k nejvzdělanějším --- 35\rm \% má vysokoškolský titul, v porovnání s 19\rm \% mileniálů.\cite{forbes2019generationX}


\section{Generace Y}\label{sec:generace-y}
Někdy označování jako \textit{mileniálové}, nebo také \textit{generace přelomu tisíciletí}, označuje populaci, jejíž představitelé se narodili v \textbf{mezi roky 1980 a 1996}. Zde jsou data poměrně flexibilní a někdy se uvádí za spodní hranici rok 1975, horním mezníkem pak dokonce rok 2005.\cite{rezlerova2007generacey} Generace baby boomu začala rodit později a jejich rodiče za ústřední princip považovali \textbf{uznávání individuality}.\cite{bergh2012coolznacky}

\section{Generace Z}\label{sec:generace-z}
Tato generace je první generací, která nezažila vyrůstání bez internetu, či nedostatku komodit. S počítačovou myší se naučili pracovat už v 18 měsících.\cite{bergh2012coolznacky} Technologie jsou pro ně samozřejmostí.
Jsou to děti generace X, kteří teprve vstupují na pracovní trh, proto o nich zatím mnoho nevíme. Je jisté, že tato generace bude technicky nejzdatnější a doposud nejdiverzitnější. Očekává se, že hlavním znakem této generace bude personalizace a kustomizace, tj. vše bude upraveno na míru.

