\chapter*{Závěr}
\addcontentsline{toc}{chapter}{Závěr}

V úvodu této práce byl marketing prezentován jako umění i věda zároveň. Tato práce se soustřeďuje zejména na vědní část. Po teoretickém úvodu do marketingu jako takového prezentuje jednotlivé generace 20. a přelomu 21. století a pomocí shromážděných dat tyto generace charakterizuje z hlediska jejich priorit, sociálního chování a chování jako spotřebitelů.
Spotřebitele rozdělujeme podle doby, ve které vyrůstaly do tzv. \textit{generací}. Tyto generace jsou specifickými zákazníky. Každá vyrůstala v jiném režimu, měla k dispozici jiné prostředky. Konformistická \textit{tichá generace} se vyměnila za generaci \textit{baby boomers}, která se postavila proti režimu. Generace \textit{X} prožila ekonomickou recesi, aby ekonomiku připravila pro individualistické \textit{mileniály}. Marketing se musí vyvíjet spolu s generacemi a s technologickým pokrokem a k tomuto vývoji je důležitá znalost chování jednotlivých generací.\\
Cílem této práce bylo charakterizovat generace X a Y, tzv. mileniály z hlediska jejich chování jako spotřebitelů. Vzhledem k tomu, že tyto generace tvoří převážnou část současného trhu práce a generace mileniálů se má stát doposud nejpočetnější generací, je znalost jejich chování pro marketing velmi cenná. Tyto generace jsou si poměrně podobné, přesto je lze postavit v několika aspektech do kontrastu.
Generace X sledují aktuální technologické trendy a myslí dopředu, zdůrazňuje rovnováhu mezi soukromým a pracovním životem a je víc než kterákoli jiná ochotna slevit ze svých mzdových nároků za cenu flexibility pracovní doby. Naproti tomu mileniálové v technologii vyrůstaly a považují ji prakticky za samozřejmou, stejně tak považují za standard flexibilní pracovní dobu a z mzdových nároků slevují jen velmi těžko. Práce je pro ně brána jako součást života, proto takový důraz na rodinný život nekladou, za to vyzvihují individualitu a inkluzivní prostředí.\\
Aby byl marketing úspěšný, je potřeba jednotlivé rozdíly zmíněné v této práci brát v úvahu a stejně tak se adaptovat na podobnosti. Např., při tvorbě značky je důležité vytvořit, tzv. \uv{cool} značku, tj. značku, se kterou se generace ztotožňují, která v nich vzbuzuje emoce, pocit sounáležitosti, autenticity a luxusu. Při její propagaci pak vytvářet reklamu, která trvá pouze nezbytně dlouho a zaujme okamžitě, protože tyto mladé generace mají velmi rozdrobenou pozornost. Dle studijí prezentovaných v této práci je také užitečné zvolit vhodně sdělovací médium --- reklamy na sociálních sítích jsou pro generace X a Y nejen nedůvěryhodné, ale někdy dokonce označeny za obtěžující --- pro maximální efektivitu a maximalizaci zisku. Monogamní náklonnost je prakticky vyloučena, ale cool značky jsou konkurenceschopnější a zvyšují šanci, že si je zákazník zvolí jako svou první volbu.