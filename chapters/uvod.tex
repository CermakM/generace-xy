\chapter*{Úvod}
\addcontentsline{toc}{chapter}{Úvod}

Marketing je o naplňování potřeb zákazníků se ziskem. Umění marketingu je uměním značky prezentovat se tak, aby vzbudila zájem, emoce, pocit sounáležitosti a luxusu. Marketing je ale také vědou, jejíž poznatky slouží k tomu, aby poskytly charakteristiku spotřebitele a umožnili tak personalizované nabídky, správnou formu a místo propagace.
Spotřebitele rozdělujeme podle doby, ve které vyrůstaly do tzv. \textit{generací}. Tyto generace jsou specifickými zákazníky. Každá vyrůstala v jiném režimu, měla k dispozici jiné prostředky. Konformistická \textit{tichá generace} se vyměnila za generaci \textit{baby boomers}, která se postavila proti režimu. Generace \textit{X} prožila ekonomickou recesi, aby ekonomiku připravila pro individualistické \textit{mileniály}. Marketing se musí vyvíjet spolu s generacemi a s technologickým pokrokem a k tomuto vývoji je důležitá znalost chování jednotlivých generací. Cílem této práce je charakterizovat jednotlivé generace z hlediska jejich chování jako spotřebitelů, zjistit jejich životní priority a analyzovat značky, které považují za \textit{cool}.