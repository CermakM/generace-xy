\chapter{Generace X a Y jako spotřebitelé}

Výzkum centra Pew Research Center zjistil, že většina příslušníků každé generace je přesvědčena o své jedinečné, charakteristické identitě\cite[s. 22]{bergh2012coolznacky}.
Marketing se zabývá zjišťováním a naplňováním lidských a společenských potřeb.\cite[s. 43]{kotler2007marketingmanagement} Aby mohl být marketing vůbec úspěšný, je klíčové tuto identitu odhalit.
Jak ukazuje obrázek č. \ref{fig:us-population-by-generation}, generace X a Y souhrnně zastávají nejpočetnější zastoupení v současné populaci. Obě tyto generace mají nicméně jiné hodnoty, rozdílné životní priority, volný čas tráví odlišným způsobem, liší se ve formě komunikace mezi sebou atd. Z pohledu marketingu se tedy chovají jinak i jako spotřebitelé.
Cílem této kapitoly je podat ucelené srovnání těchto dvou skupin jako konzumentů a zjistit, jak se chovají jako zákazníci.

\medskip
\begin{figure}[htbp!]
    \centering
    \includegraphics[width=.88\textwidth]{assets/us-population-by-generation.png}
    \caption[Populace US podle generace]{Populace US podle generace \\ Zdroj: Statista.com\protect\footnotemark~podle dat z US Census Bureau }
    \label{fig:us-population-by-generation}
\end{figure}

\footnotetext{Dostupné z: \url{https://www.statista.com/statistics/797321/us-population-by-generation/ }}

\section{Základní priority}
„Generace X byla svědkem kontrastu pracovního světa bez počítačů a příchodem technologií a digitálních inovací. Hodně porovnávají a dokážou ocenit rozsah a dopad těchto inovací, proto upřednostňují organizace, které „myslí dopředu“ a sledují aktuální technologické trendy,“ vysvětluje Ladislav Kučera v rozhovoru na téma \uv{pět generací zaměstnanců}.\cite{idnes2018generacezamestancu} Dále pak zdůrazňuje důraz generace X na rovnováhu mezi soukromým a pracovním životem - mají zodpovědnost za rodinu, děti. Tato generace je více než kterákoli jiná ochotna slevit ze svých mzdových nároků, jsou-li jí nabídnuty jiné výhody, jako je flexibilní pracovní doba, dovolená navíc, zdravotní péče a možnost pracovat z domova.

O generaci mileniálů pak personalista Ladislav Kučera prohlašuje: „Tato věková skupina je považována za zvlášť vytrvalou, ambiciózní, která se nebojí ve své kariéře trochu riskovat. Chce od svého zaměstnavatele slyšet konstruktivní zpětnou vazbu, očekává \textbf{postup v rámci své role a společnosti}“. (tamtéž)
Mileniálové nechtějí být omezováni hranicemi. Jsou si vědomi svých příležitostí a chtějí cestovat a získávat zahraniční zkušenosti.

\section{Média a sociální sítě}
Pro potřeby marketingu je nezbytně nutné šířit povědomí o značce a zvolit vhodné médium s ohledem na cílové zákazníky (reklama a místo jsou ostatně, jak bylo uvedeno, dvě ze čtyř P marketingu).
Přestože generace X se může jevit jako nihilistická a velmi \textbf{individuálně zaměřená}, podle časopisu Forbes má 81\rm \% této generace účet na Facebooku nebo jiné sociální síti\cite{forbes2019generationX} (údaj pro generaci Y sice neuvádějí, ale pravděpodobně bude přinejmenším stejně vysoké). Rozdíl ovšem je v tom, jak a jak často tyto sociální sítě využívají a jakou jim přikládají informační hodnotu.


\medskip
\begin{figure}[htbp!]
    \centering
    \includegraphics[width=.88\textwidth]{assets/gen-xy-time-spend-on-social-media.png}
    \caption[Porovnání času stráveného na sociálních sítích]{Porovnání času stráveného na sociálních sítích \\ Zdroj: The World Bank\protect\footnotemark }
    \label{fig:cas-na-socialnich-sitich}
\end{figure}

\footnotetext{Dostupné z: \url{http://web.worldbank.org/archive/website01603/WEB/MEDIA-31.HTM}}

Jak je vidět na obr. \ref{fig:cas-na-socialnich-sitich}, generace X tráví více času na médiích (včetně sociálních), než generace Y a to dokonce v řádu hodin. Toto lze vysvětlit např. způsobem, jakým mileniálové využívají svá mobilní zařízení. Analýzou více než 40 tisíc měsíčních účtů za telefon zjistila společnost Nielsen, že američtí teenageři odeslali v průměru \numprint{3146} textových zpráv za měsíc, což je asi 10 zpráv za hodinu v bdělém stavu a mimo školu.\cite[s. 33]{bergh2012coolznacky} Z dalšího výzkumu provedeného společností Deloitte\footnote{Dostupné z \url{https://blog.rescuetime.com/screen-time-stats-2018/}} vyplývá, že průměrně mileniálové kontrolují svůj telefon 58 krát za den, přičemž 70\rm \% těchto interakcí trvá méně než 2 minuty. Jiná studie dodává, že 1 z 10 teenagerů kontroluje svůj telefon alespoň jednou za 4 minuty\footnotemark.

\footnotetext{\overfullrule=0mmDostupné z \url{https://nypost.com/2017/11/08/americans-check-their-phones-80-times-a-day-study}}

Převedeno do marketingového kontextu, reklama cílená na generaci Y by měla být velmi krátká a měla by zaujmout okamžitě, jinak bude ignorována. Generace X bude klást větší důraz na její kvalitu a informační hodnotu.

Dále je velmi užitečné si uvědomit, jakou \textbf{informační hodnotu médiím jednotlivé generace přidělují}. Ve výzkumu provedeném společností eMarketer bylo zjištěno, že 82\rm \% věří tištěným reklamám nebo televizní či rádiové inzerci, zatímco pouhých 42\rm \% má důvěru ve výrobky či produkty, které jsou předmětem reklamy na sociální síti.\footnote{Dostupné z \url{https://www.emarketer.com/Article/Consumer-Trust-Evolving-Digital-Age/1014959}}


\section{Branding}
Moderní spotřebitelé jsou k novým či pro ně neznámým produktům skeptičtí a obvykle prvním krokem, než se rozhodnou pro koupi produktu, je vyhledání komentářů. Jak bylo podotknuto dříve v této práci, informace se v dnešní době šíří prakticky okamžitě a dohledat si je je velmi snadné. To dává do rukou spotřebitelů obrovskou moc a značky si tak musí být vědomy faktu, že i jedno potenciálně špatné hodnocení (byť neoprávněné) může odradit masu potenciálních zákazníků.
V průzkumu InSites Consulting zpracovali studii, ve které se dotazovaných ptali, který zdroj názorů je pro ně při rozhodování o koupi nových kusů oblečení nejdůvěryhodnější. Celkem 74\rm \% respondentů uvedlo, že je pro ně \textbf{nejdůležitější názor jejich vrstevníků}.\cite[s. 43]{bergh2012coolznacky}
Dalším negativním faktorem zejména pro mileniály pak může být neetické chování nebo špatný ekologický dopad. Mileniálové jsou uvědomělí spotřebitelé a problematika ochrany životního prostředí je stále palčivější. Demokratizaci zažívá ochrana planety obecně a pro generaci Y je typické, že se chce aktivně zapojovat a vytvářet vlastní hodnoty.

\subsection{Ztotožnění se se značkou}

Zatímco baby boomers generace si potrpěla na preciznost a kvalitu, pro generaci X a Y jsou mnohem důležitější hodnoty jako dobrá pověst (přitažlivost pro skupinu vrstevníků), kreativita a zábavnost a pravděpodobně nejdůležitější ze všech -- ztotožnění se se značkou. Značky a výrobky, které mladí kupují, se stávají \textbf{jedním z ústředních bloků, s jejichž pomocí budují svou identitu}. Vybírají si značky, se kterými se ztotožňují a které je charakterizují --- mileniálové obecně vykazují jisté formy narcismu, označovaného jako \uv{kampaň na sebe sama} --- a tím zboží dostává symbolickou hodnotu. Neslouží už jen jako nástroj a prostředek, ale zejména jako znak zastupující uvažování a hodnoty nositele.\cite{bergh2012coolznacky}

\subsection{Aakerův model}
Znalost značky sestává ze všech myšlenek, pocitů, představ, zkušeností, přesvědčení atd., které jsou spojovány se značkou. Značky musí u zákazníků především vytvářet silné, příznivé a jedinečné asociace.\cite[s. 315]{kotler2007marketingmanagement}
Příkladem může být společnost Volvo (bezbečnost), Hallmark (starostlivost), Apple (luxus).

Aaker, bývalý profesor marketingu na UC-Berkeley, pohlíží na hodnotu značky jako na soubor pěti kategorií aktiv a pasiv spojených se značkou\cite{kotler2007marketingmanagement}:
\begin{itemize}
    \item věrnost značce
    \item znalost značky
    \item vnímaná kvalita
    \item asociace spojované se značkou
    \item jiná duševní aktiva, např. patenty a obchodní známky
\end{itemize}

Jedinečný soubor asociací pak podle Aakera vytváří identitu značky sestávající se z 12 hledisek jak je ukazuje obr. \ref{fig:aaker-brand-identity-model}.

\medskip
\begin{figure}[htbp!]
    \centering
    \includegraphics[width=.95\textwidth]{assets/aaker-model.png}
    \caption[Aakerův model identity značky]{Aakerův model identity značky \\ Zdroj: vlastní}
    \label{fig:aaker-brand-identity-model}
\end{figure}

\subsection{Loayalita ke značce}
Na rozdíl od generace baby boomu a částečně i generace X, mileniálové jsou nechvalně známí pro svou přelétavost a nepředstavují skupinu, která by byla loajální ke značce. Tato generace vyrůstá ve světě plném možností a zážitků, které jsou pro ni rozdrobené do malých stimulujících okamžiků. Proto není příliš překvapivé, že je obtížné pro značku udržet si loajalitu generace Y. Dokonce i značka s nejvyšším skóre loajality, Coca-Cola, se může pochlubit jen 10\rm \% mladých, kteří její nápoj pijí jako jediné ne-alko. \uv{Monogamní} náklonnost ke značce je tedy příliš nepružný způsob zkoumání loajality.\cite{bergh2012coolznacky}

\subsection*{Coolness značky}
Jako vhodným ukazatelem loajality mileniálů a generace X ke značce se ukazuje být to, do jaké míry je pro ně značka \uv{cool}. Van der Bergh a Behrer ve své knize \textit{Jak cool značky zůstávají hot} tvrdí, že generace Y bezpochyby tvoří cílovou skupinu, která je sériově monogamní. Přecházejí mezi značkami v rámci omezené sady těch, které v jednotlivých kategoriích pokládají za důvěryhodné a cool.\cite[s. 90]{bergh2012coolznacky}

Tzv. \uv{coolness} značky lze chápat jako ukazatel toho, jako moc je značka cool. Indikátory značně závisí na produktové kategorii. Značce nicméně nesmí chybět \textit{autenticita} a \textit{exkluzivita}. Být cool znamená, že vás spotřebitelé nadšeně chtějí nakupovat.
Podle toho, na jakou věkovou kategorii marketing cílí, odlišuje cool značky také to, kolik lidí už produkt vlastní. Mladí lidé potřebují zapadnout, proto je pro ně důležitější, aby byl produkt rozšířený mezi vrstevníky. Naopak ve věku zhruba od 19 let jsou preferovány produkty, které má jen velmi málo lidí.\cite{bergh2012coolznacky}
\textbf{Exkluzivita} je samozřejmě také v inovaci, kterou produkt přináší, v neustálých vylepšeních, vývoji či variacích. V neposlední řadě jsou pak cool značky, které jsou \textbf{odlišné, individuální, originální a autentické}.\\
Graf na obr. \ref{fig:coolness-vs-loyalty} popisuje vztah mezi individuálním skóre \uv{coolness} u 540 značek ve 30 různých kategoriích a tím, zda byla konkrétní značka vybírána jako první volba.

\bigskip
\begin{figure}[htbp!]
    \centering
    \includegraphics[width=.95\textwidth]{assets/coolness-vs-loyalty.png}
    \caption[Porovnání coolness značky a loajality zákazníků]{Porovnání coolness značky a loajality zákazníků \\ Zdroj: vlastní podle \textcite[s. 95]{bergh2012coolznacky}}
    \label{fig:coolness-vs-loyalty}
\end{figure}