%%% Tento soubor obsahuje definice různých užitečných maker a prostředí %%%
%%% Další makra připisujte sem, ať nepřekáží v ostatních souborech.     %%%

\usepackage[a-2u]{pdfx}     % výsledné PDF bude ve standardu PDF/A-2u
\usepackage[top=25mm,bottom=25mm,right=25mm,left=30mm,head=12.5mm,foot=12.5mm]{geometry}
\let\openright=\cleardoublepage
 
%%% Nastavení pro použití samostatné bibliografické databáze.
\usepackage[
   autocite = superscript,
   backend  = biber,
   style    = iso-numeric,
   language = czech,
   sorting  = nty,
   bibencoding = UTF8
]{biblatex}
\addbibresource{law.bib}
\addbibresource{literatura.bib}

\DeclareCiteCommand{\supercite}[\mkbibsuperscript]
  {\iffieldundef{prenote}
     {}
     {\BibliographyWarning{Ignoring prenote argument}}%
   \iffieldundef{postnote}
     {}
     {\BibliographyWarning{Ignoring postnote argument}}}
  {\usebibmacro{citeindex}%
   \bibopenbracket\usebibmacro{cite}\bibclosebracket}
  {\supercitedelim}
  {}
  
\let\cite\supercite

\usepackage[justification=centering]{caption} % or e.g. [format=hang]

\usepackage{chngcntr}
\counterwithout{figure}{chapter}
\counterwithout{table}{chapter}

%% TODOs
\setlength{\marginparwidth}{2cm}
\usepackage[colorinlistoftodos,prependcaption,textsize=tiny]{todonotes}

%% Přepneme na českou sazbu, fonty Latin Modern a kódování češtiny
\usepackage[czech]{babel}
\usepackage{lmodern}
\usepackage[T1]{fontenc}
\usepackage{textcomp}
\usepackage[utf8]{inputenc}
\usepackage{csquotes}

%%% Další užitečné balíčky (jsou součástí běžných distribucí LaTeXu)
\usepackage{amsmath}        % rozšíření pro sazbu matematiky
\usepackage{amsfonts}       % matematické fonty
\usepackage{amsthm}         % sazba vět, definic apod.
\usepackage{bm}             % tučné symboly (příkaz \bm)
\usepackage{graphicx}       % vkládání obrázků
\usepackage{fancyvrb}       % vylepšené prostředí pro strojové písmo
\usepackage{fancyhdr}       % prostředí pohodlnější nastavení hlavy a paty stránek
\usepackage{icomma}         % inteligetní čárka v matematickém módu
\usepackage{dcolumn}        % lepší zarovnání sloupců v tabulkách
\usepackage{booktabs}       % lepší vodorovné linky v tabulkách
\makeatletter
\@ifpackageloaded{xcolor}{
   \@ifpackagewith{xcolor}{usenames}{}{\PassOptionsToPackage{usenames}{xcolor}}
  }{\usepackage[usenames]{xcolor}} % barevná sazba
\makeatother
\usepackage{multicol}       % práce s více sloupci na stránce
\usepackage{caption}
\usepackage{enumitem}
\setlist[itemize]{noitemsep, topsep=0pt, partopsep=0pt}
\setlist[enumerate]{noitemsep, topsep=0pt, partopsep=0pt}
\setlist[description]{noitemsep, topsep=0pt, partopsep=0pt}

\usepackage{float}
\usepackage{numprint}

\usepackage[acronym,nopostdot,toc]{glossaries}
\usepackage[bottom]{footmisc}

\usepackage{tabu}
\usepackage{longtable}

\usepackage{pdfpages}
\usepackage{tocloft}

\setlength\cftparskip{0pt}
\setlength\cftbeforechapskip{1.5ex}
\setlength\cftfigindent{0pt}
\setlength\cfttabindent{0pt}
\setlength\cftbeforeloftitleskip{0pt}
\setlength\cftbeforelottitleskip{0pt}
\setlength\cftbeforetoctitleskip{0pt}
\renewcommand{\cftlottitlefont}{\Huge\bfseries\sffamily}
\renewcommand{\cftloftitlefont}{\Huge\bfseries\sffamily}
\renewcommand{\cfttoctitlefont}{\Huge\bfseries\sffamily}

% vyznaceni odstavcu
\parindent=0pt
\parskip=11pt

% zakaz vdov a sirotku - jednoradkovych pocatku ci koncu odstavcu na prechodu mezi strankami
\clubpenalty=1000
\widowpenalty=1000
\displaywidowpenalty=1000

% nastaveni radkovani
\renewcommand{\baselinestretch}{1.20}

% nastaveni pro nadpisy - tucne a bezpatkove
\usepackage{sectsty}    
\allsectionsfont{\sffamily}

% nastavení hlavy a paty stránek
\fancyhf{}
\fancyhead[R]{\rightmark}
\fancyfoot[R]{\thepage}
\renewcommand{\footrulewidth}{.5pt}
\fancypagestyle{plain}{%
\fancyhf{} % clear all header and footer fields
\fancyfoot[R]{\thepage}
\renewcommand{\headrulewidth}{0pt}
\renewcommand{\footrulewidth}{0.5pt}}

% Tato makra přesvědčují mírně ošklivým trikem LaTeX, aby hlavičky kapitol
% sázel příčetněji a nevynechával nad nimi spoustu místa. Směle ignorujte.
\makeatletter
\def\@makechapterhead#1{
  {\parindent \z@ \raggedright \sffamily
   \Huge\bfseries \thechapter. #1
   \par\nobreak
   \vskip 20\p@
}}
\def\@makeschapterhead#1{
  {\parindent \z@ \raggedright \sffamily
   \Huge\bfseries #1
   \par\nobreak
   \vskip 20\p@
}}
\makeatother

% aquote pro blokovou citaci s autorem vpravo dole
\def\signed #1{{\leavevmode\unskip\nobreak\hfil\penalty50\hskip2em
  \hbox{}\nobreak\hfil(#1)%
  \parfillskip=0pt \finalhyphendemerits=0 \endgraf}}

\newsavebox\mybox
\newenvironment{aquote}[1]
  {\savebox\mybox{#1}\begin{quote}}
  {\signed{\usebox\mybox}\end{quote}}

% Trochu volnější nastavení dělení slov, než je default.
\lefthyphenmin=2
\righthyphenmin=2

% Zapne černé "slimáky" na koncích řádků, které přetekly, abychom si
% jich lépe všimli.
\overfullrule=1mm

%% Balíček hyperref, kterým jdou vyrábět klikací odkazy v PDF,
%% ale hlavně ho používáme k uložení metadat do PDF (včetně obsahu).
%% Většinu nastavítek přednastaví balíček pdfx.
\hypersetup{unicode}
\hypersetup{breaklinks=true}
\hypersetup{hidelinks}

%%% Prostředí pro sazbu kódu, případně vstupu/výstupu počítačových
%%% programů. (Vyžaduje balíček fancyvrb -- fancy verbatim.)

\DefineVerbatimEnvironment{code}{Verbatim}{fontsize=\small, frame=single}


%%% Useful law-related macros
\usepackage{pgfkeys}
\pgfkeys{
    /paragraph/.is family, /paragraph,
    default/.style = {z = 90, rok = 2012, odst = 0},
    z/.estore in = \paragraphZakon,
    rok/.estore in = \paragraphRok,
    odst/.estore in = \paragraphOdst,
}

\newcommand{\lawp}[2][]{
    \pgfkeys{/paragraph, default, #1}%
    \ifthenelse{\paragraphOdst =0}{%
        \href{https://www.zakonyprolidi.cz/cs/\paragraphRok-\paragraphZakon\#p#2}
        {\textbf{§~#2 zákona č. \paragraphZakon/\paragraphRok~Sb.}}\cite{\paragraphZakon/\paragraphRok p#2}}{%
        \href{https://www.zakonyprolidi.cz/cs/\paragraphRok-\paragraphZakon\#p#2-\paragraphOdst}
        {\textbf{§~#2 odst. \paragraphOdst~zákona č. \paragraphZakon/\paragraphRok~Sb.}}\cite{\paragraphZakon/\paragraphRok p#2-\paragraphOdst}}%
}
